%%%%%%%%%%%%%%%%%%%%%%%%%%%%%%%%%%%%%%%%%%%%%%
%%%%%%%%%%%%%%%%%%%%%%%%%%%%%%%%%%%%%%%%%%%%%%
%%                                            %%
%% IMPORTANTE                                 %%
%% -----------------------                    %%
%% ESTE FICHERO SE DEBE COMPILAR CON PDFLATEX %%
%% Estándar Latex no funciona.                %%
%%                                            %%
%%%%%%%%%%%%%%%%%%%%%%%%%%%%%%%%%%%%%%%%%%%%%%
%%%%%%%%%%%%%%%%%%%%%%%%%%%%%%%%%%%%%%%%%%%%%%

% NO MODIFICAR NADA DE LO QUE HAY A CONTINUACIÓN
%%%%%%%%%%%%%%%%%%%%%%%%%%%%%%%%%%%%%%%%%%%%%%%%%%%%%%%%%%%%%%%%%%%%%%%%%%%%%%%%%%%
% Template for Elsevier CRC journal article
% version 1.1-5p dated 18 January 2011

% This file (c) 2010-2011 Elsevier Ltd.  Modifications may be freely made,
% provided the edited file is saved under a different name

% This file contains modifications for Procedia Computer Science
% but may easily be adapted to other journals

% Changes since version 1.0
% - elsarticle class option changed from 1p to 3p (to better reflect CRC layout)
% - this version uses option 5p for larger-format journals (text area 24.1 x 18.4 cm)

%-----------------------------------------------------------------------------------

%% This template uses the elsarticle.cls document class and the extension package ecrc.sty
%% For full documentation on usage of elsarticle.cls, consult the documentation "elsdoc.pdf"
%% Further resources available at http://www.elsevier.com/latex

%-----------------------------------------------------------------------------------


%% The '5p' and 'times' class options of elsarticle are used for Elsevier CRC
\documentclass[5p,times,authoryear]{elsarticle}

%% The `ecrc' package must be called to make the CRC functionality available
%% ecrc_RIAI es el paquete ecrc de Elsevier con modificaciones para la revista RIAI
\usepackage{ecrc_RIAI}

%% The ecrc package defines commands needed for running heads and logos.
%% For running heads, you can set the journal name, the volume, the starting page and the authors

%%%%%%%%%%%%%%%%%%%%%%%%%%%%%%%%% Añadido por Secretaría RIAI
\usepackage[spanish]{babel}     % Idioma
\addto\captionsspanish{%
\def\tablename{Tabla}%
}
\usepackage[latin1]{inputenc}   % Lengua latina
\usepackage{amsmath}            % Para las referencias a ecuaciones con \eqref
\usepackage{epstopdf}           % Para poder insertar figuras .eps al compilar con PDFLATEX
\usepackage{flushend}           % Para igualar las columnas de la última página
%%%%%%%%%%%%%%%%%%%%%%%%%%%%%%%%%%%%%%%%%%%%%%%%%%%%%%%


%% set the volume if you know. Otherwise `00'
\volume{00}

%% set the starting page if not 1
\firstpage{1}

%% Give the name of the journal
\journalname{Revista Iberoamericana de Automática e Informática industrial}

%% The choice of journal logo is determined by the \jid and \jnltitlelogo commands.
%% A user-supplied logo with the name <\jid>logo.pdf will be inserted if present.
%% e.g. if \jid{yspmi} the system will look for a file yspmilogo.pdf
%% Otherwise the content of \jnltitlelogo will be set between horizontal lines as a default logo

%% Give the abbreviation of the Journal. Contast the Publisher if in doubt what this is.
\jid{RIAI}

%% Give a short journal name for the dummy logo (if needed)
\jnltitlelogo{}

%% Hereafter the template follows `elsarticle'.
%% For more details see the existing template files elsarticle-template-harv.tex and elsarticle-template-num.tex.

%% Elsevier CRC generally uses a numbered reference style
%% For this, the conventions of elsarticle-template-num.tex should be followed (included below)
%% If using BibTeX, use the style file elsarticle-num.bst

%% End of ecrc-specific commands
%%%%%%%%%%%%%%%%%%%%%%%%%%%%%%%%%%%%%%%%%%%%%%%%%%%%%%%%%%%%%%%%%%%%%%%%%%

% if you have landscape tables
\usepackage[figuresright]{rotating}

%% The amssymb package provides various useful mathematical symbols
\usepackage{amssymb}

%% The amsthm package provides extended theorem environments
%% \usepackage{amsthm}

%% The lineno packages adds line numbers. Start line numbering with
%% \begin{linenumbers}, end it with \end{linenumbers}. Or switch it on
%% for the whole article with \linenumbers after \end{frontmatter}.
%% \usepackage{lineno}

%% natbib.sty is loaded by default. However, natbib options can be
%% provided with \biboptions{...} command. Following options are
%% valid:

%%   round  -  round parentheses are used (default)
%%   square -  square brackets are used   [option]
%%   curly  -  curly braces are used      {option}
%%   angle  -  angle brackets are used    <option>
%%   semicolon  -  multiple citations separated by semi-colon
%%   colon  - same as semicolon, an earlier confusion
%%   comma  -  separated by comma
%%   numbers-  selects numerical citations
%%   super  -  numerical citations as superscripts
%%   sort   -  sorts multiple citations according to order in ref. list
%%   sort&compress   -  like sort, but also compresses numerical citations
%%   compress - compresses without sorting
%%
%% \biboptions{comma,round}

% \biboptions{}

% put your own definitions here:
%   \newcommand{\cZ}{\cal{Z}}
%   \newtheorem{def}{Definition}[section]
%   ...
% add words to TeX's hyphenation exception list
%\hyphenation{author another created financial paper re-commend-ed Post-Script}

%%%%%%%%%%%%%%%%%%%%%%%%%%%%%%%%%%%%%%%%%%%%%%%%%%%%%%%%%%
%%%%%%%%%%%%%%%%%%%%%%%%%%%%%%%%%%%%%%%%%%%%%%%%%%%%%%%%%%
%% EMPIECE A MODIFICAR A PARTIR DE AQUÍ
%%%%%%%%%%%%%%%%%%%%%%%%%%%%%%%%%%%%%%%%%%%%%%%%%%%%%%%%%%
%%%%%%%%%%%%%%%%%%%%%%%%%%%%%%%%%%%%%%%%%%%%%%%%%%%%%%%%%%

%% Escriba la lista de autores tal y como quiere que aparezca en la cabecera del artículo
%% Ejemplo \runauth{C.V. Radhakrishnan et al.}
\runauth{Primer autor et al.}

% declarations for front matter

\begin{document}

\begin{frontmatter}

%% Title, authors and addresses

%% use the tnoteref command within \title for footnotes;
%% use the tnotetext command for the associated footnote;

%% use the fnref command within \author or \address for footnotes;
%% use the fntext command for the associated footnote;

%% use the corref command within \author for corresponding author footnotes;
%% use the cortext command for the associated footnote;
%% use the ead command for the email address,
%% and the form \ead[url] for the home page:
%%
%% \title{Title\tnoteref{label1}}
%% \tnotetext[label1]{}
%% \author{Name\corref{cor1}\fnref{label2}}
%% \ead{email address}
%% \ead[url]{home page}
%% \fntext[label2]{}
%% \cortext[cor1]{}
%% \address{Address\fnref{label3}}
%% \fntext[label3]{}

%%  Use \dochead si su artículo necesita de una cabecera, e.g. \dochead{Tutorial}
%% Para la mayoría de contribuciones no es necesario utilizar \dochead
%% En RIAI normalmente se usará para los tutoriales y los benchmarks.

%% Descomentar lo que proceda...
%% \dochead{Tutorial}
%% \dochead{Benchmark}

%% Si necesita notas al pie en el título, consulte 'ejemplo latex RIAI.tex'

\title{Línea 1\\ Línea 2}

%% utilice etiquetas para ligar a los autores con las filiaciones:
%% \author[label1,label2]{<nombre autor>}
%% \address[label1]{<dirección>}
%% \address[label2]{<dirección>}
%%

%% Si necesita notas al pie para los autores, consulte 'ejemplo latex RIAI.tex'

\author[First]{Primer A. Autor \corref{cor1}}
\ead{autor1@cea-ifac.es}

% Si necesita especificar una URL para el autor, consulte 'ejemplo latex RIAI.tex'
% \ead[url]{www.autor1.es}

\author[Second]{Segundo B. Autor, Jr.}
\ead{autor2@cea-ifac.es}
\author[Third]{Tercer C. Autor}
\ead{autor3@cea-ifac.es}


\cortext[cor1]{Autor en correspondencia}

\address[First]{Filiación del primer autor.}
\address[Second]{Filiación del segundo autor.}
\address[Third]{Filiación del tercer autor.}

\begin{abstract}
%% Escriba el texto del resumen.
%% El copyright de CEA es obligatorio. 
%%%%%%%%%%%%%%%%%%%%%%%%%%%%%%%%%%%%%%
%% ATENCION: Sustituya XX por el año en curso.
%%%%%%%%%%%%%%%%%%%%%%%%%%%%%%%%%%%%%%
\emph{Copyright {\copyright} 20XX CEA.}
%%%%%%%%%%%%%%%%%%%%%%%%%%%%%%%%%%%%%%
%% ATENCION: Sustituya XX por el año en curso.
%%%%%%%%%%%%%%%%%%%%%%%%%%%%%%%%%%%%%%
\end{abstract}

\begin{keyword}
%% Escriba las palabras clave de la forma siguiente: palabra1 \sep palabra2 \sep palabra3

palabra 1 \sep palabra 2 \sep 5-10 palabras clave (tomadas de la lista del sitio web de IFAC).

\end{keyword}

\end{frontmatter}

%% Cuerpo del artículo
\section{Una sección}
% Inserte aquí el texto de la sección

\subsection{Una subsección}
% Inserte aquí el texto de la subsección. No se permite más profundidad de subsecciones (subsubsection)

\section{Otra sección}


\subsection{Otra subsección}


\subsection{Ejemplo de Tabla}

% Ejemplo de una tabla con el formato para RIAI que ocupa el ancho de la columna:

\begin{table*}[htbp]
  \caption{Escriba aquí el título para su tabla}
   \label{etiquetatabla}
  \begin{tabular}{\hsize}{lrrrrr}
\hline
    & Col 1 & Col 2 & Col 3 & Col 4 & Col 5 \\
    \hline
Fila 1 &   &  &  &  & \\
Fila 2 &  &  &  &  &  \\
Fila 3 &  &  &  &  &  \\
Fila 4 &  &  &  &  &  \\
Fila 5 &  &  &   &  & \\
Fila 6 &  &  &  &  &  \\
\hline
  \end{tabular}
\end{table*}




\section{Ejemplo de figuras}

\begin{figure}
\centering
  % Se pueden incluir figuras pdf, jpg, png al compilar con PDFLatex
  \includegraphics[width=XXXcm]{nombre_de_fichero}\\
  \caption{Título de la figura}\label{etiqueta_figura}
\end{figure}

%%%%%%%%%%%%%%%%%%%%%%%%%%%%%%%%%%%%%%%%%%%%%%%%%%%%%%%%%%%%%%%%%%%%%%%%%%%%%%%%%%%%%%%%
%% ATENCION AUTORES: ESTA SECCION ES OBLIGATORIA
%%%%%%%%%%%%%%%%%%%%%%%%%%%%%%%%%%%%%%%%%%%%%%%%%%%%%%%%%%%%%%%%%%%%%%%%%%%%%%%%%%%%%%%%

\section*{English Summary}

\textbf{Escriba aquí el título de su artículo en inglés.}\\

\noindent \textbf{Abstract}\\

% escriba aquí el resumen en inglés de su artículo

\\

\noindent \emph{Keywords:}\\

% Escriba aquí las palabras clave en inglés.

%%%%%%%%%%%%%%%%%%%%%%%%%%%%%%%%%%%%%%%%%%%%%%%%%%%%%%%%%%%%%%%%%%%%%%%%%%%%%%%%%%%%%%%%%%

\section*{Agradecimientos}

% Escriba aquí el texto para los agradecimientos. Esta sección NO es obligatoria



%% REFERENCIAS.
%% Este estilo para las referencias es obligatorio.
\bibliographystyle{elsarticle-harv}


\bibliography{}

%% Se recomienda usar una base de datos BibTeX para las referencias.
%% Puede insertar sus referencias sin un fichero BibTex así:

% \begin{thebibliography}{00}

%% \bibitem must have the following form:
%%   \bibitem{key}...
%%

% \bibitem{}

% \end{thebibliography}


% Si su artículo tiene apéndices, insértelos de la siguiente forma:

\appendix
\section{Primer Apéndice}    % Capa Apéndice debe tener un título corto.
% Escriba aquí el texto correspondiente al apéndice.

\section{Segundo Apéndice}
% Escriba aquí el texto correspondiente al apéndice.


\end{document}

%%
%% End of file `plantilla latex RIAI.tex'.