% Template for Elsevier CRC journal article
% version 1.1-5p dated 18 January 2011
% SEM ACENTUACAO PROBLEMA UTF-8 : NAO ACEITA

% This file (c) 2010-2011 Elsevier Ltd.  Modifications may be freely made,
% provided the edited file is saved under a different name

% This file contains modifications for Procedia Computer Science
% but may easily be adapted to other journals

% Changes since version 1.0
% - elsarticle class option changed from 1p to 3p (to better reflect CRC layout)
% - this version uses option 5p for larger-format journals (text area 24.1 x 18.4 cm)

%-----------------------------------------------------------------------------------

%% This template uses the elsarticle.cls document class and the extension package ecrc.sty
%% For full documentation on usage of elsarticle.cls, consult the documentation "elsdoc.pdf"
%% Further resources available at http://www.elsevier.com/latex

%-----------------------------------------------------------------------------------

%%%%%%%%%%%%%%%%%%%%%%%%%%%%%%%%%%%%%%%%%%%%%%
%%%%%%%%%%%%%%%%%%%%%%%%%%%%%%%%%%%%%%%%%%%%%%
%%                                          %%
%% Important note on usage                  %%
%% -----------------------                  %%
%% This file must be compiled with PDFLaTeX %%
%% Using standard LaTeX will not work!      %%
%%                                          %%
%%%%%%%%%%%%%%%%%%%%%%%%%%%%%%%%%%%%%%%%%%%%%%
%%%%%%%%%%%%%%%%%%%%%%%%%%%%%%%%%%%%%%%%%%%%%%

%% The '5p' and 'times' class options of elsarticle are used for Elsevier CRC
\documentclass[5p,times,authoryear]{elsarticle}

%% The `ecrc' package must be called to make the CRC functionality available
%% ecrc_RIAI es el paquete ecrc de Elsevier con modificaciones para la revista RIAI
\usepackage{ecrc_RIAI}

%% The ecrc package defines commands needed for running heads and logos.
%% For running heads, you can set the journal name, the volume, the starting page and the authors

%%%%%%%%%%%%%%%%%%%%%%%%%%%%%%%%% Aaadido por Secretaraa RIAI
\usepackage[spanish]{babel}     % Idioma
\addto\captionsspanish{%
\def\tablename{Tabla}%
}
\usepackage[latin1]{inputenc}   % Lengua latina
\usepackage{amsmath}            % Para las referencias a ecuaciones con \eqref

\usepackage{epstopdf}           % Para poder insertar figuras .eps al compilar con PDFLATEX
\usepackage{flushend}           % Para igualar las columnas de la altima pagina
\usepackage{natbib}
%\usepackage{hyperref}           % Para hipervanculos dentro del PDF
%%%%%%%%%%%%%%%%%%%%%%%%%%%%%%%%%%%%%%%%%%%%%%%%%%%%%%%


%% set the volume if you know. Otherwise `00'
\volume{00}

%% set the starting page if not 1
\firstpage{1}

%% Give the name of the journal
\journalname{Revista Internacional de An\'alisis Financiero}

%% Give the author list to appear in the running head
%% Example \runauth{C.V. Radhakrishnan et al.}
\runauth{María Luisa Argáez Salcido }

%% The choice of journal logo is determined by the \jid and \jnltitlelogo commands.
%% A user-supplied logo with the name <\jid>logo.pdf will be inserted if present.
%% e.g. if \jid{yspmi} the system will look for a file yspmilogo.pdf
%% Otherwise the content of \jnltitlelogo will be set between horizontal lines as a default logo

%% Give the abbreviation of the Journal. Contast the Publisher if in doubt what this is.
\jid{RIAI}

%% Give a short journal name for the dummy logo (if needed)
\jnltitlelogo{}

%% Hereafter the template follows `elsarticle'.
%% For more details see the existing template files elsarticle-template-harv.tex and elsarticle-template-num.tex.

%% Elsevier CRC generally uses a numbered reference style
%% For this, the conventions of elsarticle-template-num.tex should be followed (included below)
%% If using BibTeX, use the style file elsarticle-num.bst

%% End of ecrc-specific commands
%%%%%%%%%%%%%%%%%%%%%%%%%%%%%%%%%%%%%%%%%%%%%%%%%%%%%%%%%%%%%%%%%%%%%%%%%%

%% The amssymb package provides various useful mathematical symbols
\usepackage{amssymb}
%% The amsthm package provides extended theorem environments
%% \usepackage{amsthm}

%% The lineno packages adds line numbers. Start line numbering with
%% \begin{linenumbers}, end it with \end{linenumbers}. Or switch it on
%% for the whole article with \linenumbers after \end{frontmatter}.
%% \usepackage{lineno}

%% natbib.sty is loaded by default. However, natbib options can be
%% provided with \biboptions{...} command. Following options are
%% valid:

%%   round  -  round parentheses are used (default)
%%   square -  square brackets are used   [option]
%%   curly  -  curly braces are used      {option}
%%   angle  -  angle brackets are used    <option>
%%   semicolon  -  multiple citations separated by semi-colon
%%   colon  - same as semicolon, an earlier confusion
%%   comma  -  separated by comma
%%   numbers-  selects numerical citations
%%   super  -  numerical citations as superscripts
%%   sort   -  sorts multiple citations according to order in ref. list
%%   sort&compress   -  like sort, but also compresses numerical citations
%%   compress - compresses without sorting
%%
%% \biboptions{comma,round}

% \biboptions{}

% if you have landscape tables
\usepackage[figuresright]{rotating}

% put your own definitions here:
%   \newcommand{\cZ}{\cal{Z}}
%   \newtheorem{def}{Definition}[section]
%   ...

% add words to TeX's hyphenation exception list
%\hyphenation{author another created financial paper re-commend-ed Post-Script}

% para poder introducir varias figuras que ocupen el ancho de las dos columnas.
\usepackage{subfigure}

% declarations for front matter

\begin{document}

\begin{frontmatter}

%% Title, authors and addresses

%% use the tnoteref command within \title for footnotes;
%% use the tnotetext command for the associated footnote;

%% use the fnref command within \author or \address for footnotes;
%% use the fntext command for the associated footnote;

%% use the corref command within \author for corresponding author footnotes;
%% use the cortext command for the associated footnote;
%% use the ead command for the email address,
%% and the form \ead[url] for the home page:
%%
%% \title{Title\tnoteref{label1}}
%% \tnotetext[label1]{}
%% \author{Name\corref{cor1}\fnref{label2}}
%% \ead{email address}
%% \ead[url]{home page}
%% \fntext[label2]{}
%% \cortext[cor1]{}
%% \address{Address\fnref{label3}}
%% \fntext[label3]{}

%\dochead{Cabecera artaculo}
%% Use \dochead if there is an article header, e.g. \dochead{Short communication}

\title{Segmentaci\'on de retornos de la bola de valores de S $\&$ P 500}


%% use optional labels to link authors explicitly to addresses:
%% \author[label1,label2]{<author name>}
%% \address[label1]{<address>}
%% \address[label2]{<address>}

\author[First]{Mar\'ia Arg\'aez Autor\corref{cor1}\fnref{label2}}
\ead{margaezs@uanl.edu.mx}
\ead[url]{www.cea-ifac.es}

 

\fntext[label2]{Ing. Matem\'atica por UACH y estudiante en la maestr\'ia de Ciencia de Datos por la UANL}
\cortext[UANL]{Universidad Aut\'onoma de Nuevo Le\'on}


\address[First]{Pedro de Alba S/N, Ni\~{n}os H\'eroes, Ciudad Universitaria, San Nicol\'as de los Garza, N.L. }
 

\begin{abstract}
%% Text of abstract
Este art\'iculo presenta un an\'alisis de datos que involucra un an\'alisis exploratorio de datos desde medidas descriptivas hasta segmentaci\'on de datos utilizando algoritmos de aprendizaje autom\'atico no supervisado como el K-means.  \emph{Copyright {\copyright} XXXX CEA. Publicado por Elsevier Espaaa, S.L. Todos los derechos reservados.}
\end{abstract}

\begin{keyword}
%% keywords here, in the form: keyword \sep keyword

%% MSC codes here, in the form: \MSC code \sep code
%% or \MSC[2008] code \sep code (2000 is the default)
Aprendizaje aut\'omatico no supervisado \sep an\'alisis de datos  \sep S\& P500    .

\end{keyword}

\end{frontmatter}

%%
%% Start line numbering here if you want
%%
% \linenumbers

%% main text
\section{Introducci\'on}

El an\'alisis exploratorio de datos brinda una idea general del comportamiento de los datos estad\'isticamente, en donde se utilizan medidas centrales y de dispersi\'on como la media, la mediana, desviaci\'on estandar, varianza y rango.

% https://www.mjandrews.org/book/ddsr/assets/chapter_05_exploratory.pdf

La agrupaci\'on en cl\'usteres o segmentaci\'on, es una t\'ecnica computacional que divide las muestras de conjunto de datos en grupos. Una agrupaci\'on exitosa da como resultado grupos que contienen puntos que est\'an relacionados entre s\'i. Si esas relaciones son significativas generalmente requiere verificaci\'on humana.
% https://livebook.manning.com/book/classic-computer-science-problems-in-python/chapter-6/1



\section{Material y m\'etodos}
Los datos que se utilizaron fueron 5 a\~{n}os de historia diaria de la bolsa de valores llamada "S \& P 500", la cual cuenta con valores como el 'Open '  (Apertura), ' Close' (cierre), 'High' (alto), 'Low' (bajo). Estas medidas de la serie de tiempo con medidas  refleja el precio m\'inimo y m\'aximo de la bolsa asi como el precio con el que abrio y cerro. 

Se calcularon los retornos con la serie de tiempo del cierre de la acción, lo cu\'al es la divisón del valor anterior del cierre entre el actual.

Se conoce que existen diversos tipos bolsa de valores, en este caso "S $\&$ P 500", son las siglas en ingl\'es de "Standard and Poor's 500", el cu\'al es un \'indice burs\'atil que rastrea el rendimiento de las acciones de 500 grandes empresas que cotizan en las bolsas de valores de los Estados Unidos y representa uno de los \'indices burs\'atiles m\'as seguidos.

Para este an\'alisis se utilizo el software libre de python y librerias de sklearn donde se utiliz\'o el algoritmo de K - means.

%%\subsection{Insercian de tablas}

% La tabla ocupa el ancho de la columna porque el entorno \emph{tabular} lleva el asterisco. Se puede usar \emph{table}* para confeccionar una tabla que se expanda sobre la dos columnas del texto. Y por supuesto combinar ambos efectos. \citep{Heritage:92}, \citep{ChaRou:66}


% \begin{table}[htbp]
%   \caption{Preferencias para el diseao de un controlador}
%    \label{extremos45}
%   \begin{tabular*}{\hsize}{lrrrrr}
% \hline
%     & $g_i^1$ & $g_i^2$ & $g_i^3$ & $g_i^4$ & $g_i^5$ \\
%     \hline
% $Re(\lambda)_{max}$ & -0.01  & -0.005 & -0.001 & -0.0005 & -0.0001 \\
% $u_{max}$& 0.85 & 0.90 & 1 & 1.5 & 2  \\
% $t_{est}^{max}$& 14 & 16 & 18 & 21 & 25 \\
% $noise_{max}$& 0.5 & 0.9 & 1.2 & 1.4 & 1.5  \\
% $u_{nom}$& 0.5 & 0.7 & 1  & 1.5 & 2  \\
% $t_{est}^{nom}$& 10 & 11 & 12 & 14 & 15 \\
% \hline
%   \end{tabular*}
% \end{table}
 
 

\section{Experimento}



\section{C\'alculos}


\section{Resultados}

 

\section{Conclusi\'on}

 
%%%%%%%%%%%%%%%%%%%%%%%%%%%%%%%%%%%%%%%%%%%%%%%%%%%%%%%%%%%%%%%%%%%%%%%%%%%%%%%%%%%%%%%%
%% ATENCION AUTORES: ESTA SECCION ES OBLIGATORIA
%%%%%%%%%%%%%%%%%%%%%%%%%%%%%%%%%%%%%%%%%%%%%%%%%%%%%%%%%%%%%%%%%%%%%%%%%%%%%%%%%%%%%%%%
 
%%%%%%%%%%%%%%%%%%%%%%%%%%%%%%%%%%%%%%%%%%%%%%%%%%%%%%%%%%%%%%%%%%%%%%%%%%%%%%%%%%%%%%%%%%
 

%% The Appendices part is started with the command \appendix;
%% appendix sections are then done as normal sections
%% \appendix

%% \section{}
%% \label{}

%% References
%%
%% Following citation commands can be used in the body text:
%% Usage of \cite is as follows:
%%   \cite{key}         ==>>  [#]
%%   \cite[chap. 2]{key} ==>> [#, chap. 2]
%%


%% References with BibTeX database:

\bibliographystyle{elsarticle-harv}
%%\bibliography{riaibib}

%% Authors are advised to use a BibTeX database file for their reference list.
%% The provided style file elsarticle-num.bst formats references in the required Procedia style

%% For references without a BibTeX database:

%%\begin{thebibliography}{01}

%% \bibitem must have the following form:
%%   \bibitem{key}...
%%

% \bibitem{}

%%\bibitem[{Able(1945)}]{Abl:45}
%%Able, B., 1945. Nombre del artículo. Nombre de la revista 35, 123--126.

\begin{thebibliography}{7}

\expandafter\ifx\csname natexlab\endcsname\relax\def\natexlab#1{#1}\fi
\expandafter\ifx\csname url\endcsname\relax
  \def\url#1{\texttt{#1}}\fi
\expandafter\ifx\csname doi\endcsname\relax
  \def\doi#1{\texttt{#1}}\fi
\expandafter\ifx\csname urlprefix\endcsname\relax\def\urlprefix{URL: }\fi
\expandafter\ifx\csname doiprefix\endcsname\relax\def\doiprefix{DOI: }\fi

\bibitem[{Able(1945)}]{Abl:45}
Able, B., 1945. Nombre del art\'iculo. Nombre de la revista 35, 123--126.
\newline\doiprefix\doi{10.3923/ijbc.2010.190.202}
 

 \end{thebibliography}
 

\end{document}

%%
%% End of file `ejemplo latex RIAI.tex'.